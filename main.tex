\documentclass[5p,sort&compress]{elsarticle}	
\newcommand{\RomanNumeralCaps}[1]
    {\MakeUppercase{\romannumeral #1}}
\makeatletter
\def\ps@pprintTitle{%
 \let\@oddhead\@empty
 \let\@evenhead\@empty
 \def\@oddfoot{\footnotesize\itshape
       \hfill\today}%
 \let\@evenfoot\@oddfoot}
\makeatother
\usepackage{dcolumn}
\usepackage[utf8]{inputenc}
\usepackage[T1]{fontenc}

\usepackage{textcomp}
\usepackage{lmodern}
\renewenvironment{abstract}{\global\setbox\absbox=\vbox\bgroup
\hsize=\textwidth\def\baselinestretch{1}%
\noindent\unskip\textbf{Abstract}
\par\medskip\noindent\unskip\ignorespaces}
{\egroup}
\usepackage{amsmath}
\usepackage{amssymb}
\usepackage{caption} %% figurer fugler
\usepackage{bm}
\usepackage{siunitx}
\sisetup{
exponent-product = \cdot,
output-decimal-marker  =  {.}, % komma-stil
separate-uncertainty = false, %true hvis du ikke vil ha usikkerhet i parentes
per-mode = symbol,
group-digits = false,
}
\usepackage{graphicx}
\renewcommand{\topfraction}{.85}
\renewcommand{\bottomfraction}{.7}
\renewcommand{\textfraction}{.15}
\renewcommand{\floatpagefraction}{.66}
\setcounter{topnumber}{3}
\setcounter{bottomnumber}{2}
\setcounter{totalnumber}{10}
\usepackage{flafter}
\usepackage{booktabs}
\usepackage{multirow}
\usepackage{hyperref}
\hypersetup{
    colorlinks=true,
    linkcolor=blue,
    filecolor=magenta,      
    urlcolor=cyan,
}
\usepackage{cleveref}
\usepackage{comment}
\usepackage{arydshln}
\usepackage[font=small,labelfont=bf]{caption}
\usepackage{subcaption}

% Dakrmode
\usepackage{xcolor}
\pagecolor[rgb]{0.128, 0.128, 0.128}  %black
\color[rgb]{0.848, 0.848, 0.848}  %grey

\urlstyle{same}


%%% LINK TIL DRIVE MED DATA

% https://drive.google.com/drive/folders/1--5J3TUoHqHPiAaSORvcn1TaKvSFUdGK?usp=sharing

%%%

\begin{document}
\begin{frontmatter}

  \title{TFE4575: Chemical methods for thin film deposition}
  %\title{Characterization of an Unknown Sample Using Scanning Electron Microscopy, Scanning Transmission Electron Microscopy, Energy Dispersive X-Ray Spectroscopy, Atomic Force Microscopy, and Raman Spectroscopy}

  \author[fysikk]{Brynjar Morka Mæhlum}
  \author[fysikk]{Thord Niri Gjesdhal Heggren}


  \address[fysikk]{Department of Physics, Norwegian University of Science and Technology, 7491 Trondheim, Norway.}

  \begin{abstract}

    \noindent So abstract, wow!

  \end{abstract}


\end{frontmatter}

{ % Denne gjør table of contents pink i stedet for blå
\hypersetup{linkcolor=pink}
\tableofcontents
}

%%%%%%%%%%%%%%%%%%% THEORY %%%%%%%%%%%%%%%%%%
\section{Theory}

\subsection{Sol-Gel Synthesis Method}
\noindent This subsection is based on chapter 3 of B. L. Cushing \textit{et al.} review paper \textit{Recent Advances in the Liquid- Phase Synthesis of Inorganic Nanoparticles} \cite{solgel_review}.

In general, sol-gel processing combines small molecules to form a solid material. This is done using a solution of precursors (the \textit{sol}) that forms a network of bound molecules (the \textit{gel}). Traditionally, sol-gel processing only referred to the hydrolysis and condensation of alkoxide based precoursors such as Si(OEt)$_2$ (tetraethyl orthosilicate), but today it refers to all processes using sol-gel. The sol-gel synthesis method can be divided into the following six distinct steps. 

Step (1): The formation of a stable solution of the alkoxide or solvated metal precursor. 

Step (2): The gelation that results in the formation  an oxide- or alcohol-bridged network by polycondensation or polyesterification reactions. This dramatically increases the the viscosity of the solution. 

Step (3):  The aging of the cell, also known as syneresis. In this step, the gel network contracts and expulses the solution from the pores, and the reactions continue until the gel forms a solid mass.

Step (4): The drying of the gel where water and other volatile liquids are removed. This step is complicated because it fundamentally changes the gel structure. The drying process comprises of four sub-steps: (i) the constant rate period, (ii) the critical point, (iii), the first falling rate period, and (iv) the second falling rate period. The result is either termed a xerogel, if isolated by thermal evaporation, or an aerogel, if the solvent is extracted under supercritical conditions. 

Step (5): Dehydration of the gel using high temperatures. This removes surface-bound M-OH groups, thus stabilizing against rehydration.

Step (6): The densification and decomposition of the gel. This makes the gel pores collapse, and all remaining organic species are volatilized. 

\subsection{Chemistry of the Sols}

\noindent Using the sol-gel synthesis method, one can produce thin films. One example of such a film is BTO (Barium titanium oxygen), which can be made of a mixture of barium sol and titanium sol. The following paragraphs explains the components of these sols and what their functions are.

The barium sol can be made of a mixture of water, EDTA (Ethylenediaminetetraacetic acid), ammonia solution, barium nitrate, and citric acid. 

\subsection{Equipment Used}

\subsubsection{SEM}

Scanning electron microscopes (SEM) are used for sample analysis, and

%%%%%%%%%%%%%%%%%%% METHODS %%%%%%%%%%%%%%%%%%
\section{Methods}
\subsection{Solution preparation}

\noindent First, the barium solution was made.
In a beaker, 10 mL of DI water was heated to 60°C while being stirred with a magnet.
% TODO: legg til RPM på stirring. 
% brynjar: det er ikke skrevet ned i notatene.
To this, 3.6893 g of EDTA and 7 mL of ammonia solution was added, and the solution was stirred until clear.
Then, 3.2676 g of Ba(NO$_3$)$_2$ and 4.8070 g of citric acid was added, and the solution was again stirred until clear.
Finally, about 3.5 mL of ammonia solution was added to achieve a pH of 7.
This was verified using pH paper.
The final solution was then added to a 50 mL volumetric flask, and the volume adjusted to 50 mL with DI water.


This titanium solution used in this project was premade by NTNU NanoLab.
However, the processing steps are included here for completeness and ease of replication.
In a beaker, 50 mL of DI water was heated to 60°C while being stirred with a magnet.
To this, 14.409 g of citric acid was added.
Then, 7.6 mL of titanium isopropoxide was added with a syringe.
The solution was then covered with parafilm and stirred overnight, until the solution was clear.
Finally, ammonia solution was added to adjust the pH to 7.
The solution was then added to a bottle, and standardized to 0.501 mmol/gram using DI water.

% TODO, kommentar: (fikset, men les over)
% It should be clear for the reader why these amounts are used. Remember that the volume of Ti sol is irrelevant here, as it is standardized per unit weight. Thus, you should rather say how much Ti-sol you weighted out, and that you added Ba-sol in a such way that the Ba-Ti ratio of the ions was 1:1 (assuming that Barium Nitrate was completely dissolved).
% Gammel tekst:
% Finally, to make the BTO solution, 3 mL of titanium sol was first added to a beaker. 
% Then, 6.90 mL of barium sol was added, and the solutions were mixed together.
% Nyt tekst:
Since the titanium sol is standardized per unit weight, the volume is irrelevant, but the amount of titanium sol is important.
3.4420 g of titanium sol was weighed out, and added to a beaker.
The amount of barium sol needed to match that Ba-Ti ratio of the ions was 1:1.
Thus, 6.90 mL of barium sol was added, and the solutions were mixed together. % Vi la til 4.0+2.9 mL
It is here assumed that the barium nitrate got completely dissolved.


\subsection{Substrate coating}

\noindent The BTO was deposited on a Si-Pt substrate.
Before the deposition, the substrate was cleaned with an ISO 5 clean room wipe, submerged in acetone, and rinsed with isopraoanol and water.
Then, the wafer was dried with nitrogen gas, and placed in the Plasma Cleaner from Diener Electronics for 2 minutes with oxygen flow and 50/50 O$_2$/generator power.
The substrate was then placed in the spin coater and secured using vacuum.
Using a syringe with a 0.2 \textmu m filter, the BTO was applied to the substrate.
The spinning was done at 500 rpm for 10 seconds, then 2500 rpm for 30 seconds, with 5 seconds acceleration time.
After the deposition, the substrate backside was cleaned using an ISO 5 clean room wipe and acetone.
Then, the substrate was baked on a hot plate at 200°C for about 4 minutes.
% TODO kommentar (fikset, men les over):
% Heating and cooling as fast as possible is a rather arbitrary description. The heating rate is a very important parameter for solvent evaporation and crystallization, and you should write it in the units of C/min. The cooling rate is not equally important, but it can have an influence. For example, the domain size of Erbium Manganite, that you presented in figure 2, will depend on cooling rate when cooled from above the Curie temperature.
% Gammel setning: Finally, sintering was done in a gold furnace at 700°C for 10 minutes, heating and cooling as fast as possible.
% Ny setning
Sintering was done in a gold furnace at 700°C for 10 minutes, with a heating rate of 35°C/minute. % har fra notatene at den ble varmet opp i 20 min, men det var det som klikka. Synes vi ikke skal skrive noe om det.
The substrate was cooled down to room temperature in the furnace.


\subsection{Characterization}

\noindent First, the thin film was inspected using an optical microscope.
This was done at both 5X and 20X magnification.
Then, the film was inspected using a Dektak 150 profilometer.
This was done at 3 different locations on the sample, and the profiles were taken from the edge of the sample to the center.
The instrument was set to measure hills and valleys for 4000 microns in 60 seconds, with a resolution of 0.222 \textmu m per sample.

% SEM
Finally, the film was inspected using the SEM APREO.
SEM images were taken on low voltage with a goal of imaging both the film surface and possibly the ferroelectric domain structure.
The acceleration voltage was varied between 2 and 5 kV, and the beam current was varied between 50 and 200 pA.
The magnification was 200X, 5000X, and 12000X.
Working distance was set to what the software recommended for each set of settings.
The detectors used were the ETD detector for overview and the Trinity 2 (T2) secondary electron detector for the high resolution images.



%%%%%%%%%%%%%%%%%%% RESULTS %%%%%%%%%%%%%%%%%%
\section{Results}

resul

These are the results. Results are good. fgfdgd

\noindent Best results ever, am I right?!

\subsection{Optical Images}

\subsection{Profilometer}

For the profilometer analysis, you should perform this at three different places on the sample. Make profiles from the edge of the sample and to the center.

\subsection{Scanning Electron Microscope}

Results should also include SEM and optical images of both samples. It may be that you need to coat the samples with a conductive coating (i.e. carbon or gold) for SEM imaging as barium titanate is not very conductive. Comment on whether or not you did this in your report. Since coating might be necessary, it is recommended that you do the profilometer and the optical analysis first, as coating will affect the roughness and the thickness of the sample.

At least two magnifications of each sample (for SEM and optical microscope) should be shown in the report. It is important to get one overview image, which shows the overall morphology of the film; and one closer image where you only focus on a small section to look at grain size, close-up of cracks and other potential surface artifacts.

%%%%%%%%%%%%%%%%%%% DISCUSSION %%%%%%%%%%%%%%%%%%
\section{Discussion}

\noindent I think this. No this is much better. Every discussion ever.

%%%%%%%%%%%%%%%%%%% CONCLUSION %%%%%%%%%%%%%%%%%%
\section{Conclusion}

\noindent To conclude, this is boring!





%%%%%%%%%%%%%% REFERANSER %%%%%%%%%%%%%%%%%%

\begingroup
\begin{center}
  \rule{2cm}{.4pt}
\end{center}
\makeatletter
\@beginparpenalty=10000
\makeatother
\bibliographystyle{unsrt} %unsrtnat
\bibliography{references}


\endgroup

\end{document}