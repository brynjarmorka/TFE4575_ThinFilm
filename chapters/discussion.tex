
\subsection{Optical Images}

\noindent \autoref{fig:optical_overview} shows that visually, away from the substrate edges, the film appears to be homogenous.
The blue areas at the bottom and to the right could be due to a different film thickness in these regions.
Different thicknesses will reflect light differently and thus be another color.
The reason for this deviation might be due to bad spin coating or some edge effect during the heating steps.

In the same figure, there is a large, circular artifact.
Visually, this looks like the film has flaked off the substrate. 
It is difficult to extract the exact reason why this happened.
One suggestion is that the adhesion between the film and the substrate was insufficient due to bad cleaning.
However, if this was the case, one could probably expect this to happen in other areas as well.

\autoref{fig:optical_20x} shows that artifacts were present at the regions that appeared homogeneously in the lower magnification image.
From the picture, it looks like there are two types of artifacts -  one blue circular and several black smudges.
The difference between them might be that the black ones lies \textit{on} the film, while the blue lays \textit{in} the film.
The reason for this difference may be related to when these artifacts arrived to the sample, where the blue ones are from before the film was dry.
It is possible that the black ones are a result of the bad sample storage.
However, with the information available from only optical images, it is not possible to conclude on what these artifacts are and why they are there.

\subsection{Profilometer}

\noindent In general, the results from the profilometer shows that the film is very flat.

% flattness and roughness. STD is RMS which kinda is roughness

% potential pores

% artifacts. peak and deformation

\subsection{SEM}

% roughness in SE images.

%  morphology of the surface

% surface cracked or smooth

% edge effect and corner

% artifacts

% pores we might see

% compare flaked corner with optical image. What we see in SEM and not in optical


% which section gives best morphology results