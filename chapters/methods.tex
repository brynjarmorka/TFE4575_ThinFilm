\subsection{Barium sol}

In a beaker, 10 mL of DI water was heated to 60°C while being stirred with a magnet.
To this, 3.6893 g of EDTA and 7 mL of ammonia solution was added, and the solution was stirred until clear.
Then, 3.2676 g of Ba(NO$_3$)$_2$ and 4.8070 g of citric acid was added, and the solution was again stirred until clear.
Finally, about 3.5 mL of ammonia solution was added to achieve a pH of 7. 
This was verified using pH paper.
The final solution was then added to a 50 mL volumetric flask, and the volume adjusted to 50 mL with DI water.

\subsection{Titanium sol}

This titanium solution used in this project was premade by NTNU NanoLab.
However, the processing steps are included here for completeness and ease of replication.

In a beaker, 50 mL of DI water was heated to 60°C while being stirred with a magnet.
To this, 14.409 g of citric acid was added.
Then, 7.6 mL of titanium isopropoxide was added with a syringe.
The solution was then covered with parafilm and stirred overnight, until the solution was clear.
Finally, ammonia solution (30 \%) was added to adjust the pH to 7.

\subsection{Barium-titanium sol}

To make the barium-titanium sol (BTO