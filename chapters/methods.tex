\subsection{Solution preparation}

\noindent First, the barium solution was made.
In a beaker, 10 mL of DI water was heated to 60°C while being stirred with a magnet.
% TODO: legg til RPM på stirring. 
% brynjar: det er ikke skrevet ned i notatene.
% niri: har heller ikke peiling på RPM, kan enten bare skrive noe eller kanskje drite i det... 
To this, 3.6893 g of EDTA and 7 mL of ammonia solution was added, and the solution was stirred until clear.
Then, 3.2676 g of Ba(NO$_3$)$_2$ and 4.8070 g of citric acid was added, and the solution was again stirred until clear.
Finally, about 3.5 mL of ammonia solution was added to achieve a pH of 7.
This was verified using pH paper.
The final solution was then added to a 50 mL volumetric flask, and the volume adjusted to 50 mL with DI water.


This titanium solution used in this project was premade by NTNU NanoLab.
However, the processing steps are included here for completeness and ease of replication.
In a beaker, 50 mL of DI water was heated to 60°C while being stirred with a magnet.
To this, 14.409 g of citric acid was added.
Then, 7.6 mL of titanium isopropoxide was added with a syringe.
The solution was then covered with parafilm and stirred overnight, until the solution was clear.
Finally, ammonia solution was added to adjust the pH to 7.
The solution was then added to a bottle, and standardized to 0.501 mmol/gram using DI water.

% TODO, kommentar: (fikset, men les over)
% tenker det er good
% It should be clear for the reader why these amounts are used. Remember that the volume of Ti sol is irrelevant here, as it is standardized per unit weight. Thus, you should rather say how much Ti-sol you weighted out, and that you added Ba-sol in a such way that the Ba-Ti ratio of the ions was 1:1 (assuming that Barium Nitrate was completely dissolved).
% Gammel tekst:
% Finally, to make the BTO solution, 3 mL of titanium sol was first added to a beaker. 
% Then, 6.90 mL of barium sol was added, and the solutions were mixed together.
% Nyt tekst:
Since the titanium sol is standardized per unit weight, the volume is irrelevant, but the amount of titanium sol is important.
3.4420 g of titanium sol was weighed out, and added to a beaker.
The amount of barium sol needed to match that Ba-Ti ratio of the ions was 1:1.
Thus, 6.90 mL of barium sol was added, and the solutions were mixed together. % Vi la til 4.0+2.9 mL
It is here assumed that the barium nitrate got completely dissolved.


\subsection{Substrate coating}

\noindent The BTO was deposited on a Si-Pt substrate.
Before the deposition, the substrate was cleaned with an ISO 5 clean room wipe, submerged in acetone, and rinsed with isopraoanol and water.
Then, the wafer was dried with nitrogen gas, and placed in the Plasma Cleaner from Diener Electronics for 2 minutes with oxygen flow and 50/50 O$_2$/generator power.
The substrate was then placed in the spin coater and secured using vacuum.
Using a syringe with a 0.2 \textmu m filter, the BTO was applied to the substrate.
The spinning was done at 500 rpm for 10 seconds, then 2500 rpm for 30 seconds, with 5 seconds acceleration time.
After the deposition, the substrate backside was cleaned using an ISO 5 clean room wipe and acetone.
Then, the substrate was baked on a hot plate at 200°C for about 4 minutes.
% TODO kommentar (fikset, men les over):
% Heating and cooling as fast as possible is a rather arbitrary description. The heating rate is a very important parameter for solvent evaporation and crystallization, and you should write it in the units of C/min. The cooling rate is not equally important, but it can have an influence. For example, the domain size of Erbium Manganite, that you presented in figure 2, will depend on cooling rate when cooled from above the Curie temperature.
% Gammel setning: Finally, sintering was done in a gold furnace at 700°C for 10 minutes, heating and cooling as fast as possible.
% Ny setning
Sintering was done in a gold furnace at 700°C for 10 minutes, with a heating rate of 35°C/minute. % har fra notatene at den ble varmet opp i 20 min, men det var det som klikka. Synes vi ikke skal skrive noe om det.
The substrate was cooled down to room temperature in the furnace.


\subsection{Characterization}

\noindent First, the thin film was inspected using an optical microscope.
This was done at both 5X and 20X magnification.
Then, the film was inspected using a Dektak 150 profilometer.
This was done at 3 different locations on the sample, and the profiles were taken from the edge of the sample to the center.
The instrument was set to measure hills and valleys for 4000 microns in 60 seconds, with a resolution of 0.222 \textmu m per sample.

% SEM
Finally, the film was inspected using the SEM APREO.
SEM images were taken on low voltage with a goal of imaging both the film surface and possibly the ferroelectric domain structure.
The acceleration voltage was varied between 2 and 5 kV, and the beam current was varied between 50 and 200 pA.
The magnification was 200X, 5000X, and 12000X.
Working distance was set to what the software recommended for each set of settings.
The detectors used were the ETD detector for overview and the Trinity 2 (T2) secondary electron detector for the high resolution images.