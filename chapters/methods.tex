\subsection{Solution preparation}

First, the barium solution was made.
In a beaker, 10 mL of DI water was heated to 60°C while being stirred with a magnet.
To this, 3.6893 g of EDTA and 7 mL of ammonia solution was added, and the solution was stirred until clear.
Then, 3.2676 g of Ba(NO$_3$)$_2$ and 4.8070 g of citric acid was added, and the solution was again stirred until clear.
Finally, about 3.5 mL of ammonia solution was added to achieve a pH of 7. 
This was verified using pH paper.
The final solution was then added to a 50 mL volumetric flask, and the volume adjusted to 50 mL with DI water.


This titanium solution used in this project was premade by NTNU NanoLab.
However, the processing steps are included here for completeness and ease of replication.
In a beaker, 50 mL of DI water was heated to 60°C while being stirred with a magnet.
To this, 14.409 g of citric acid was added.
Then, 7.6 mL of titanium isopropoxide was added with a syringe.
The solution was then covered with parafilm and stirred overnight, until the solution was clear.
Finally, ammonia solution (30 \%) was added to adjust the pH to 7.
The solution was then added to a bottle, and standardized to 0.501 mmol/gram using DI water.

Finally, to make the barium titanium oxygen (BTO) sol, 3 mL of titanium sol was first added to a beaker. 
Then, 6.90 mL of barium sol was added, and the solutions were mixed together.

\subsection{Substrate coating}

The BTO was deposited on a Si-Pt substrate.
Before the deposition, the substrate was cleaned with an ISO 5 clean room wipe, submerged in acetone, and rinsed with isopraoanol and water.
Then, the wafer was dried with nitrogen gas, and placed in a plasma cleaner for 2 minutes with oxygen flow and 50/50 O$_2$/generator power.
The substrate was then placed in the spin coater and secured using vacuum.
Using a syringe with a 0.2 \textmu m filter, the BTO sol was applied to the substrate.
The spinning was done at 500 rpm for 10 seconds, then 2500 rpm for 30 seconds, with 5 seconds acceleration time.
After the deposition, the substrate backside was cleaned using an ISO 5 clean room wipe and acetone.
Then, the substrate was baked on a hot plate at 200°C for about 4 minutes.
Finally, sintering was done in a gold furnace at 700°C for 10 minutes, heating and cooling as fast as possible.

\subsection{Characterization}

