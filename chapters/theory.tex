\subsection{Sol-Gel Synthesis Method}
\noindent This subsection is based on chapter 3 of B. L. Cushing \textit{et al.} review paper \textit{Recent Advances in the Liquid- Phase Synthesis of Inorganic Nanoparticles} \cite{solgel_review}.

In general, sol-gel processing combines small molecules to form a solid material. This is done using a solution of precursors (the \textit{sol}) that forms a network of bound molecules (the \textit{gel}). Traditionally, sol-gel processing only referred to the hydrolysis and condensation of alkoxide based precoursors such as Si(OEt)$_2$ (tetraethyl orthosilicate), but today it refers to all processes using sol-gel. The sol-gel synthesis method can be divided into the following six distinct steps. 

Step (1): The formation of a stable solution of the alkoxide or solvated metal precursor. 

Step (2): The gelation that results in the formation  an oxide- or alcohol-bridged network by polycondensation or polyesterification reactions. This dramatically increases the the viscosity of the solution. 

Step (3):  The aging of the cell, also known as syneresis. In this step, the gel network contracts and expulses the solution from the pores, and the reactions continue until the gel forms a solid mass.

Step (4): The drying of the gel where water and other volatile liquids are removed. This step is complicated because it fundamentally changes the gel structure. The drying process comprises of four sub-steps: (i) the constant rate period, (ii) the critical point, (iii), the first falling rate period, and (iv) the second falling rate period. The result is either termed a xerogel, if isolated by thermal evaporation, or an aerogel, if the solvent is extracted under supercritical conditions. 

Step (5): Dehydration of the gel using high temperatures. This removes surface-bound M-OH groups, thus stabilizing against rehydration.

Step (6): The densification and decomposition of the gel. This makes the gel pores collapse, and all remaining organic species are volatilized. 

\subsection{Chemistry of the Sols}

\noindent Using the sol-gel synthesis method, one can produce thin films. One example of such a film is BTO (Barium titanium oxygen), which can be made of a mixture of barium sol and titanium sol. The following paragraphs explains the components of these sols and what their functions are.

The barium sol can be made of a mixture of water, EDTA (Ethylenediaminetetraacetic acid), ammonia solution, barium nitrate, and citric acid. 

\subsection{Equipment Used}

\subsubsection{SEM}

Scanning electron microscopes (SEM) are used for sample analysis, and