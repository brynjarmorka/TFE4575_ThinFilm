% results


% These are the results. Results are good.

% \noindent Best results ever, am I right?!

% \subsection{Optical Images}

% \subsection{Profilometer}

% For the profilometer analysis, you should perform this at three different places on the sample. Make profiles from the edge of the sample and to the center.

% \subsection{Scanning Electron Microscope}

% Results should also include SEM and optical images of both samples. It may be that you need to coat the samples with a conductive coating (i.e. carbon or gold) for SEM imaging as barium titanate is not very conductive. Comment on whether or not you did this in your report. Since coating might be necessary, it is recommended that you do the profilometer and the optical analysis first, as coating will affect the roughness and the thickness of the sample.

% At least two magnifications of each sample (for SEM and optical microscope) should be shown in the report. It is important to get one overview image, which shows the overall morphology of the film; and one closer image where you only focus on a small section to look at grain size, close-up of cracks and other potential surface artifacts.


% optical inspection


% profilometer


% SEM
\section{SEM images}
\label{results:SEM}

SEM images were acquired on the SEM APREO at NTNU NanoLab, without coating of the sample.
One of the engineers at NanoLab suggested that the sample would be conductive enough, and showed examples of SEM results from other ferroelectric thin films that were not coated, e.g. \cite{hunnestad_visualizing_2019}.
The examples used low voltage and low current, which was also used in this work.


% SEM image of the corner
Figure \cref{fig:sem:corner} shows an overview SEM image of a corner, which is the same area in the optical image in \cref{fig:optical:corner}.
The image was taken with 3 kV and 50 pA, using the EDT detector to get both topography and z-contrast.
The scale bar is 500 $\mu$m.


% SEM image of what might be pores
Figure \cref{fig:sem:pores} shows a closer SEM image of a part of the sample.
The image was taken with 5 kV and 0.1 nA, using the T2 SE detector.


